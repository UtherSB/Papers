\documentclass[12pt]{article}
\usepackage{amsmath}
\usepackage{amsthm}
\usepackage{amssymb}
\usepackage{graphicx}
\usepackage{multicol}
\usepackage[dvips,letterpaper]{geometry} % Margins.
\usepackage{hyperref}
\usepackage{comment}
\usepackage{bbm}

%\makeatletter % Need for anything that contains an @ command 
%\renewcommand{\maketitle} % Redefine maketitle to conserve space
%{ \begingroup \vskip 10pt \begin{center} \LARGE {\emph \@title}
%	\vskip 40pt \large \@author \vskip 10pt \@date \end{center}
% \vskip 10pt \endgroup \setcounter{footnote}{0} }
%\makeatother % End of region containing @ commands

\let\underdot=\d % rename builtin command \d{} to \underdot{}
\renewcommand{\d}[2]{\frac{d #1}{d #2}} % for derivatives
\newcommand{\dd}[2]{\frac{d^2 #1}{d #2^2}} % for double derivatives
\newcommand{\pd}[2]{\frac{\partial #1}{\partial #2}} % for partial derivatives
\newcommand{\pdd}[2]{\frac{\partial^2 #1}{\partial #2^2}} % for double partial derivatives
\newcommand{\pdc}[3]{\left( \frac{\partial #1}{\partial #2}\right)_{#3}} % for thermodynamic partial derivatives
\newcommand{\avg}[1]{\langle #1 \rangle}
\newcommand{\ket}[1]{| #1 \rangle}
\newcommand{\bra}[1]{\langle #1 |}
\newcommand{\braket}[2]{\langle #1 | #2 \rangle} % for Dirac brackets
\newcommand{\ketbra}[2]{| #1 \rangle\langle #2 |}
\newcommand{\matrixel}[3]{\langle #1 | #2 | #3 \rangle} % for Dirac matrix elements

\def\dbar{{\mathchar'26\mkern-12mu d}}

\newtheorem{theorem}{Theorem}
\newtheorem{proposition}[theorem]{Proposition}
\newtheorem{lemma}[theorem]{Lemma}
\theoremstyle{plain}
\newtheorem{definition}[theorem]{Definition}
\theoremstyle{remark}
\newtheorem*{remark}{Remark}
\theoremstyle{plain}
\newtheorem{corollary}[theorem]{Corollary}
\newtheorem*{example}{Example}

\DeclareMathOperator{\SL}{SL}
\DeclareMathOperator{\SO}{SO}
\DeclareMathOperator{\SU}{SU}
\DeclareMathOperator{\Spin}{Spin}
\DeclareMathOperator{\Oh}{O}
\DeclareMathOperator{\eS}{S}

\usepackage{bbm}
\usepackage{enumerate}
\newcommand{\behaviour}{(\Xi,\widetilde{\mathcal{M}}, P)}
\DeclareMathOperator{\proj}{proj}
\DeclareMathOperator{\st}{s.t.}
\DeclareMathOperator{\Span}{span}
\DeclareMathOperator{\Tr}{Tr}

\begin{document}
\title{Controllability of a system evolving under the symplectic group $Sp(2n,\mathbb{R})$}
\maketitle

One can recast the evolution of a system evolving under the action of a quadratic Hamiltonian with $2n$ continuous degrees of freedom in the symplectic representation:

\begin{equation}
\d{S}{t} = A(t)S \quad S(0) = \mathbbm{1}.
\end{equation}

The $A(t)$ are determined by the control functions. By manipulating the control functions let's assume we can obtain a set

\begin{equation}
\mathcal{E} = \{ \tilde{A}_1, \ldots, \tilde{A}_m\}
\end{equation}
of linearly independent generators of a Lie algebra $\mathcal{L}$, where $\mathcal{G} = e^{\mathcal{L}}$. As a result, any $K \in \mathcal{G}$ can be written
\begin{equation}
K = e^{\tilde{A}_1t_1}e^{\tilde{A}_2t_2}\ldots e^{\tilde{A}_mt_m} \quad \text{with } \tilde{A}_j \in \mathcal{E} \text{ and } t_j  \in \mathbb{R}
\end{equation}

If the group $G$ is compact then for $t < 0$ there exists a sequence of positive times $t_k >0$ such that 
\begin{equation}
\lim_{k \to \infty} e^{\tilde{A}t_k} = e^{\tilde{A}t}
\end{equation}
Hence we can generate any element of $\mathcal{G}$ using only positive times and we call the system controllable.

\textbf{The above is directly taken from the 2012 Dynamical recurrence paper. Do you mean $\lim_{t_k \to \infty}$? Otherwise $\lim_{k \to \infty} t_k$ where the sequence converges is just some finite time and the system is exactly periodic.} \\

If $\mathcal{G}$ is noncompact then we look for a condition on $\tilde{A}$ such that for $t_1 > 0$ there exists a $t_2 > 0$ such that 
\begin{equation}
e^{\tilde{A}(-t_1)} = e^{\tilde{A}t_2}
\end{equation}
which would act as an extra `compact-like' condition on our generators so that we are able to claim that the group is controllable.

This is equivalent to a periodicity condition on $\tilde{A}$ as:
\begin{equation}
\begin{aligned}
&e^{\tilde{A}(-t_1)} = e^{\tilde{A}t_2} \\
\implies &e^{\tilde{A}(t_1+t_2)} = \mathbbm{1} \\
\implies &e^{\tilde{A}2(t_1+t_2)} = \mathbbm{1}
\end{aligned}
\end{equation}
So $e^{\tilde{A}t}$ has period of $T = t_1 + t_2$ in this case.

For the constant matrix $\tilde{A}$, periodicity is equivalent to it having pure imaginary eigenvalues and being diagonalisable. (as shown in the other pdf).

\section{Almost periodicity}
The statement that one only requires almost periodicity is equivalent to one only needing a transformation arbitrarily close to the symplectic transformation required.

For $g'$ in the set of all possibly reachable transformations with positive time and $g \in Sp(2n,\mathbb{R})$ we require
\begin{equation}
\exists g' \text{ such that } ||g' - g|| < \epsilon
\end{equation}
for any given $\epsilon$ (presumably set by the experimenter).

This sounds fine as a definition of controllability but is certainly broader than that defined by d'Alessandro and the Lie Algebra Rank Criterion.

\end{document}