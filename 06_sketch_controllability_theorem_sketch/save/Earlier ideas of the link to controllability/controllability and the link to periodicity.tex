\include{header}

\begin{document}
\title{Controllability of a system evolving under the symplectic group $Sp(2n,\mathbb{R})$}
\maketitle

One can recast the evolution of a system evolving under the action of a quadratic Hamiltonian with $2n$ continuous degrees of freedom in the symplectic representation:

\begin{equation}
\d{S}{t} = A(t)S \quad S(0) = \mathbbm{1}.
\end{equation}

The $A(t)$ are determined by the control functions. By manipulating the control functions let's assume we can obtain a set

\begin{equation}
\mathcal{E} = \{ \tilde{A}_1, \ldots, \tilde{A}_m\}
\end{equation}
of linearly independent generators of a Lie algebra $\mathcal{L}$, where $\mathcal{G} = e^{\mathcal{L}}$. As a result, any $K \in \mathcal{G}$ can be written
\begin{equation}
K = e^{\tilde{A}_1t_1}e^{\tilde{A}_2t_2}\ldots e^{\tilde{A}_mt_m} \quad \text{with } \tilde{A}_j \in \mathcal{E} \text{ and } t_j  \in \mathbb{R}
\end{equation}

If the group $G$ is compact then for $t < 0$ there exists a sequence of positive times $t_k >0$ such that 
\begin{equation}
\lim_{k \to \infty} e^{\tilde{A}t_k} = e^{\tilde{A}t}
\end{equation}
Hence we can generate any element of $\mathcal{G}$ using only positive times and we call the system controllable.

\textbf{The above is directly taken from the 2012 Dynamical recurrence paper. Do you mean $\lim_{t_k \to \infty}$? Otherwise $\lim_{k \to \infty} t_k$ where the sequence converges is just some finite time and the system is exactly periodic.} \\

If $\mathcal{G}$ is noncompact then we look for a condition on $\tilde{A}$ such that for $t_1 > 0$ there exists a $t_2 > 0$ such that 
\begin{equation}
e^{\tilde{A}(-t_1)} = e^{\tilde{A}t_2}
\end{equation}
which would act as an extra `compact-like' condition on our generators so that we are able to claim that the group is controllable.

This is equivalent to a periodicity condition on $\tilde{A}$ as:
\begin{equation}
\begin{aligned}
&e^{\tilde{A}(-t_1)} = e^{\tilde{A}t_2} \\
\implies &e^{\tilde{A}(t_1+t_2)} = \mathbbm{1} \\
\implies &e^{\tilde{A}2(t_1+t_2)} = \mathbbm{1}
\end{aligned}
\end{equation}
So $e^{\tilde{A}t}$ has period of $T = t_1 + t_2$ in this case.

For the constant matrix $\tilde{A}$, periodicity is equivalent to it having pure imaginary eigenvalues and being diagonalisable. (as shown in the other pdf).

\section{Almost periodicity}
The statement that one only requires almost periodicity is equivalent to one only needing a transformation arbitrarily close to the symplectic transformation required.

For $g'$ in the set of all possibly reachable transformations with positive time and $g \in Sp(2n,\mathbb{R})$ we require
\begin{equation}
\exists g' \text{ such that } ||g' - g|| < \epsilon
\end{equation}
for any given $\epsilon$ (presumably set by the experimenter).

This sounds fine as a definition of controllability but is certainly broader than that defined by d'Alessandro and the Lie Algebra Rank Criterion.

\end{document}