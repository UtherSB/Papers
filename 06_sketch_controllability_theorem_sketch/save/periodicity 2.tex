\documentclass[12pt]{article}
\usepackage{amsmath}
\usepackage{amsthm}
\usepackage{amssymb}
\usepackage{graphicx}
\usepackage{multicol}
\usepackage[dvips,letterpaper]{geometry} % Margins.
\usepackage{hyperref}
\usepackage{comment}
\usepackage{bbm}

%\makeatletter % Need for anything that contains an @ command 
%\renewcommand{\maketitle} % Redefine maketitle to conserve space
%{ \begingroup \vskip 10pt \begin{center} \LARGE {\emph \@title}
%	\vskip 40pt \large \@author \vskip 10pt \@date \end{center}
% \vskip 10pt \endgroup \setcounter{footnote}{0} }
%\makeatother % End of region containing @ commands

\let\underdot=\d % rename builtin command \d{} to \underdot{}
\renewcommand{\d}[2]{\frac{d #1}{d #2}} % for derivatives
\newcommand{\dd}[2]{\frac{d^2 #1}{d #2^2}} % for double derivatives
\newcommand{\pd}[2]{\frac{\partial #1}{\partial #2}} % for partial derivatives
\newcommand{\pdd}[2]{\frac{\partial^2 #1}{\partial #2^2}} % for double partial derivatives
\newcommand{\pdc}[3]{\left( \frac{\partial #1}{\partial #2}\right)_{#3}} % for thermodynamic partial derivatives
\newcommand{\avg}[1]{\langle #1 \rangle}
\newcommand{\ket}[1]{| #1 \rangle}
\newcommand{\bra}[1]{\langle #1 |}
\newcommand{\braket}[2]{\langle #1 | #2 \rangle} % for Dirac brackets
\newcommand{\ketbra}[2]{| #1 \rangle\langle #2 |}
\newcommand{\matrixel}[3]{\langle #1 | #2 | #3 \rangle} % for Dirac matrix elements

\def\dbar{{\mathchar'26\mkern-12mu d}}

\newtheorem{theorem}{Theorem}
\newtheorem{proposition}[theorem]{Proposition}
\newtheorem{lemma}[theorem]{Lemma}
\theoremstyle{plain}
\newtheorem{definition}[theorem]{Definition}
\theoremstyle{remark}
\newtheorem*{remark}{Remark}
\theoremstyle{plain}
\newtheorem{corollary}[theorem]{Corollary}
\newtheorem*{example}{Example}

\DeclareMathOperator{\SL}{SL}
\DeclareMathOperator{\SO}{SO}
\DeclareMathOperator{\SU}{SU}
\DeclareMathOperator{\Spin}{Spin}
\DeclareMathOperator{\Oh}{O}
\DeclareMathOperator{\eS}{S}

\usepackage{bbm}
\usepackage{enumerate}
\newcommand{\behaviour}{(\Xi,\widetilde{\mathcal{M}}, P)}
\DeclareMathOperator{\proj}{proj}
\DeclareMathOperator{\st}{s.t.}
\DeclareMathOperator{\Span}{span}
\DeclareMathOperator{\Tr}{Tr}
\DeclareMathOperator{\Real}{Re}
\usepackage{tikz}
\usetikzlibrary{fit,matrix}

\begin{document}
\title{Conditions on the Hamiltonian for periodic symplectic transformations}
\maketitle

\section{Notation}\label{sec:notation}
\begin{equation}
\Omega = \bigoplus_{i=1}^n \omega \quad  \text{where} \quad \omega = \begin{pmatrix} 0 & 1 \\ -1 & 0 \end{pmatrix}
\end{equation}

We consider the Hamiltonian for a system $\hat{H} = \frac{1}{2}z^T H z$ where
\begin{equation} z = (\hat{x}_1, \hat{p_1}, \ldots, \hat{x}_n, \hat{p}_n)^T. \end{equation}
and where
\begin{equation} [\hat{x}_i,\hat{p}_j] = i\Omega_{ij}. \end{equation}

The corresponding evolution equation can be shown to be
\begin{equation} \dot{z} = Az \end{equation} where $A \in sp(2n,\mathbb{R})$ and $A = \Omega H$.\\

Later we will have the case of a diagonal $2n \times 2n$ matrix. Each of the $n$, $2 \times 2$ subblocks starting from the first and working down in jumps of two will be called `2-blocks' as shown by the boxed submatrices of matrix M:

\begin{equation*}
  M=
  \tikz[baseline=(M.west)]{%
    \node[matrix of math nodes,matrix anchor=west,left delimiter=(,right delimiter=),ampersand replacement=\&] (M) {%
      a \& b \& 0 \& 0 \& \cdots \\
      c \& d \& 0 \& 0 \& \cdots \\
      0 \& 0 \& f \& g \& \cdots \\
      0 \& 0 \& h \& i \& \cdots \\
      \vdots \& \vdots \& \vdots \& \vdots \& \ddots \\
    };
    \node[draw,fit=(M-1-1)(M-2-2),inner sep=-1pt] {};
    \node[draw,fit=(M-3-3)(M-4-4),inner sep=-1pt] {};
  }
\end{equation*}





%T%H%E%O%R%E%M%S%

\section{Theorem - Periodicity's link to A}
% Eigenvalues of a log matrix.
\begin{lemma} The eigenvalues of the logarithm of a matrix are the logarithms of the eigenvalues of the original matrix. \label{lem:eigenlog} \end{lemma} 
\begin{proof}
Sugggested in Yakubovich. I would like to get a proper verification of this. SORT OUT.
\end{proof}

% A diag iff eA diag.
\begin{lemma} A matrix, M, is diagonalisable iff $e^M$ is diagonalisable. \label{lem:eadiag} \end{lemma} 
\begin{proof}
Stated on internet - (see saved forum chat in favourite bar.)
\end{proof}

% Theorem 1
\begin{theorem} Periodicity is equivalent to A being pure imaginary and diagonalsable. \label{lem:periodicity} \end{theorem}
\begin{proof}
Periodicity is equivalent to the fact that $\exists$ T such that:
\begin{equation}
e^{A(t+T)}=e^{At} \implies e^{AT}=\mathbb{I}
\end{equation}
\begin{itemize}
\item This implies that the eigenvalues of A must be pure imaginary or zero by lemma (\ref{lem:eigenlog}). 
\item The identity matrix is diagonal and therefore A must be diagonalisable by lemma (\ref{lem:eadiag}).
\end{itemize}
\end{proof}

%%%%%%%%%%%%%% BREAK IN THEOREMS %%%%%%%%%%%%%%%%%%%%

\section{Theorem - A's link to H}
% Positive negative eigenvalues.
\begin{lemma} The characteristic polynomial of a real Hamiltonian matrix is an even polynomial. Thus, if $\lambda$ is an eigenvalue of a Hamiltonian matrix, then so are $-\lambda, \bar{\lambda}, -\bar{\lambda}$. \label{lem:hamevalues} \end{lemma}
\begin{proof}
See p45 Meyer. 
\end{proof}

% Symplectic A diagonalisation.
\begin{lemma} A is diagonalisable, or at least block diagonalisable by a symplectic transformation.
\end{lemma}
\begin{proof}
NEED TO SORT THIS OUT.
\end{proof}

% A 2x2 to real form.
\begin{lemma} Let A be a real $2 \times 2$ Hamiltonian matrix with eigenvalues $\pm \beta i$, $\beta \neq 0$. There exists a real $2 \times 2$ symplectic matrix $S$ such that
\begin{equation}
S^{-1}AS = \begin{pmatrix} 0 & \beta \\ -\beta & 0 \end{pmatrix} \quad \text{or} \quad S^{-1}AS = \begin{pmatrix} 0 & -\beta \\ \beta & 0 \end{pmatrix}
\end{equation}
\label{lem:reala} \end{lemma}
\begin{proof}
see Meyer p60 2nd ed.
\end{proof}

% A to H
\begin{lemma}Consider the $2 \times 2$ matrix $A$ such that $A=\Omega H$, the eigenvalues of $A$ are $\pm i \beta$ and $H$ is real and symmetric. This implies that the eigenvalues of $H$ are positive or negative definite. \label{lem:atoh}
\end{lemma}
\begin{proof}
\begin{equation}
H = \begin{pmatrix} a & b \\ c & d \end{pmatrix} \implies A = \begin{pmatrix} 0 & 1 \\ -1 & 0 \end{pmatrix} \begin{pmatrix} a & b \\ c & d \end{pmatrix} = \begin{pmatrix} c & d \\ -a & -b \end{pmatrix}
\end{equation}
so we know that
\begin{equation}
-bc+ad = \beta^2 \quad and \quad c-b=0
\end{equation}
using the two eigenvalue identities given in the proof above. One can check that these conditions are consistent and imply that the eigenvalues of $A$ are $\pm i \beta$.

Now write the characteristic equation for $H$:
\begin{equation}
\begin{aligned}
(a-\lambda)(d-\lambda) - bc &= \lambda^2-(a+d)\lambda +ad - bc = \lambda^2 - (a+d)\lambda + \beta^2, \\
\lambda &= \frac{1}{2}\left(a+d \pm \sqrt{(a+d)^2-4\beta^2}\right).
\end{aligned}
\end{equation}
We know that $(a+d)^2 \geq 4\beta^2$ because the eigenvalues must be real.

We also know that
\begin{equation}
| \sqrt{(a+d)^2 - 4\beta^2} | < |(a+d)|
\end{equation}
remembering that $a$ and $d$ may be negative.

As a result the only two options are that $H$ is positive or negative definite.
\end{proof}

% Not A, not definite H.
\begin{lemma}
Consider the $2 \times 2$ matrix $A$ such that $A=\Omega H$, the eigenvalues of $A$ are $\pm \beta$, $\beta > 0$ and $H$ is real and symmetric. This implies that the eigenvalues of $H$ are not definite. \label{lem:hnotdefinite}
\end{lemma}
\begin{proof}
\begin{equation}
H = -\begin{pmatrix}0 & 1 \\ -1 & 0 \end{pmatrix} \begin{pmatrix} a & b \\ c & d \end{pmatrix} = \begin{pmatrix} -c & -d \\ a & b \end{pmatrix}
\end{equation}
such that 
\begin{equation}
ad - cb = -\mu^2 \quad \text{and} \quad a + d = 0 \quad \text{such that } \mu \in \mathbb{R}^+
\end{equation} 
The characteristic equation of $H$ is
\begin{equation}
\begin{aligned}
(-\lambda - c)(-\lambda + b) + ad &= \lambda^2 + (c - b)\lambda - bc + ad = 0. \\
& = \lambda^2 + (c-b)\lambda - \mu^2 = 0
\end{aligned}
\end{equation}

\begin{equation}
\begin{aligned}
\lambda &= \frac{1}{2}(-(c-b) \pm \sqrt{(c-b)^2 + 4\mu^2}) \\
\lambda &= \frac{1}{2}((b-c) \pm \sqrt{(b-c)^2 + 4\mu^2}).
\end{aligned}
\end{equation}
\begin{equation}
\sqrt{(b-c)^2 + 4\mu^2} > b-c
\end{equation}
and hence this matrix is not definite.
\end{proof}

% 2n2n
\begin{lemma}Let $A$ be a $2n \times 2n$ Hamiltonian matrix with distinct eigenvalues 
\begin{equation} \lambda_1, \ldots \lambda_n, -\lambda_1, \ldots  -\lambda_n; \end{equation} then there exists a symplectic matrix $S$ (possibly complex) such that 
\begin{equation}
S^{-1}AS = diag(\lambda_1, \ldots, \lambda_n,-\lambda_1, \ldots -\lambda_n).
\end{equation}
\label{lem:2n2n}
\end{lemma} 
\begin{proof}
See p46 of Meyer.
\end{proof}

% Thereom 2
\begin{theorem}$A$ is pure imaginary and diagonalisable iff $H$ is $2\times2$ block diagonalisable and positive definite, negative definite, or 0 on its 2-blocks.
\end{theorem}
\begin{proof}
\textbf{Pure imaginary implies:} \\
A is Hamiltonian so lemma (\ref{lem:hamevalues}) ensures that these pure imaginary eigenvalues occur in $\pm$ pairs. This also applies to the $0$ eigenvalues to keep the matrix even dimensional.

Enact two linear symplectic transformations. First we diagonalise A, which is possible by lemma (?????) IS THIS LINEAR AND SYMPLECTIC?, into its $\pm$ pairs and then put it into one of the real forms from lemma (\ref{lem:reala}).

Lemma (\ref{lem:atoh}) implies that the corresponding Hamiltonian is positive definite, negative definite, or 0 on each of its 2-blocks.\\

\textbf{Now it remains to show that if A is not pure imaginary then $H$ is not positive definite, negative definite or 0 on its 2-blocks.}

\begin{enumerate}
\item \textit{A is diagonalisable but not pure imaginary.} In its diagonal form each of the 2-blocks of A must be symplectic and so they cannot be truly complex by lemma (\ref{lem:hamevalues}), therefore they must be real (and we assume non zero). If they are real then eigenvalues of the H 2-blocks are not definite by lemma (\ref{lem:hnotdefinite}.
\item \textit{A is pure imaginary but not diagonalisable.} H is $2 \times 2$ block diagonalisable, therefore so is $A$. Therefore the eigenvalues of the 2-blocks of A are $\pm i\beta$ for some $\beta \in \mathbb{R}$. This implies that the 2-blocks are diagonalisable by lemma (\ref{lem:2n2n}). This is a contradiction.
\item \textit{A is not pure imaginary and not diagonalisable.} H is $2 \times 2$ block diagonalisable, therefore so is $A$. Therefore the eigenvalues of the 2-blocks of A are $\pm \beta$ for some $\beta \in \mathbb{R}$. This implies that the 2-blocks are diagonalisable by lemma (\ref{lem:2n2n}). This is a contradiction.
\end{enumerate}

Proving that X implies Y and `not X' implies `not Y' proves that X $\iff$ Y.
\end{proof}

\section{Notes}
H is not, in general 2block diagable.

\section{A different method}
\begin{equation}
A = JH
\end{equation}
A is Hamiltonian therefore completely reducible. The subblocks are Hamiltonian and so
\begin{equation}
A = \bigoplus_{j=1}^s A_j \quad \text{where } A_j = JH_j
\end{equation}


$H_j$ is Hamiltonian so there exists $S \in Sp(2n,\mathbb{R} \text{ or } \mathbb{C}?)$ such that 
\begin{equation}
S^TH_jS = D_{H_j}
\end{equation}
where $D_{H_j}$ is real and diagonal. So
\begin{equation}
S^{-1}A_jS = A_j' = S^{-1}JH_jS = JS^TH_jS = JD_{H_j}
\end{equation}
By lemma (on desktop) $A_j'$ is still Hamiltonian.
\begin{equation}
D_{H_j} = -JA_j' = -\begin{pmatrix} 0 & I \\ -I & 0 \end{pmatrix}\begin{pmatrix}a & b \\ c & d \end{pmatrix} = \begin{pmatrix} -c & -d \\ a & b \end{pmatrix}
\end{equation}
where $a, b, c, d$ are $n_j \times n_j$ matrices, where $2n_j = \text{dim}\mathbb{V}_j$. Therefore $a = 0$, $d = 0$ and $b$ and $c$ are both real and diagonal. Therefore
\begin{equation}
A_j' = \begin{pmatrix} 0 & b \\ c & 0 \end{pmatrix}
\end{equation}
where $b$ and $c$ are diagonal matrices. The elements of $b$ and $-c$ are the diagonal elements of $D_{H_j}$.\\

\textbf{No assumptions have been made so far about the properties of $A_j$ other than that it is Hamiltonian}.\\
In our direct sum decomposition we can ensure that each $A_j$ has only a pair of real, pair of imaginary, or quadruple of truly complex, eigenvalues.

\begin{lemma}
If $D_{H_j}$ is definite then $A_j'$ is pure imaginary and diagonalisable.
\end{lemma}
\begin{proof}
Proved in Genoni2012. Double check no explicit use of $\Omega$.
\end{proof}
\begin{proposition}
If $A_j'$ has eigenvalues $\pm i\beta$, $\beta \in \mathbb{R}$, and is diagonalisable then $b$ and $c$ are both definite and in the opposite sense. $b = \text{diag}(\beta,\beta,\ldots)$ and $c = \text{diag}(-\beta,-\beta,\ldots)$.
\end{proposition}
\begin{proof}
Rephrased below.
\end{proof}

\textbf{Rephrasing everything:}\\
\begin{equation}
A = \begin{pmatrix} 0 & B \\ -C & 0 \end{pmatrix}
\end{equation}
A is a pure imaginary, diagonalisable, $2n \times 2n$ Hamiltonian matrix with eigenvalues $\pm i\beta$. $B$ and $C$ are both $n \times n$ diagonal matrices. The spectrum of $A$ contains $n$ lots of $+i\beta$ and $n$ lots of $-i\beta$. $\text{det}(A) = \beta^{2n}$.
\begin{proposition}
$B = \text{diag}(b, b,\ldots)$ and $C = \text{diag}(c, c, \ldots)$ where $b, c \in \mathbb{R}^+$. $\beta = \sqrt{bc}$. 
\end{proposition}
\begin{proof}
To do.
\end{proof}
In this case this would imply that the eigenvalues of $H_j$ were $b$ and $c$.

The other proposition becomes:
\begin{proposition}
If $B$ and $C$ are both definite in the same sense then A is purely imaginary. An extension of this down to $B$ and $C$ being made up of 2 values of element because haven't got further with the other one. Although this may not be helpful for this proof.
\end{proposition}

\subsection{Lemma for the prop}
To prove above just need:
\begin{equation} M = \begin{pmatrix} 0 & A \\ B & 0 \end{pmatrix} \end{equation}
\begin{lemma}
The eigenvalues of $M$ are $\pm \sqrt{a1b1},\pm \sqrt{a2b2},\ldots$.
\end{lemma}
\begin{proof}
From wiki.
\begin{equation}
\text{det}\begin{pmatrix} A & B \\ C & D \end{pmatrix} = \text{det}(AD-BC)
\end{equation}
If $CD=DC$ and $A, B, C, D$ are of the same order.
\end{proof}



\subsection{Theorem}
\begin{theorem}
A is diagonalisable with pure imaginary eigenvalues is equivalent to $H$ being PDNDZ on its submatrices $H_j$, defined above.
\end{theorem}|
\end{document}


%%%%%%%% END OF DOCUMENT %%%%%%%%%%%















%%%%%%%%%%%%%%%% unneeded.
\begin{lemma} Any symplectic matrix A with eigenvalues $\pm i\beta$ can be put in the form... by similarity transform. \label{lem:reala} \end{lemma}
\begin{proof}
see Meyer. FILL IN FROM MEYER.
\end{proof}

%%%%%%%%%% Real and symmetric diagonalisation.
\begin{lemma} Any real and symmetric matrix is diagonalisable by a similarity transformation, i.e. a change of basis.  \label{lem:realsymm}
\end{lemma}
\begin{proof}
Real symmetric matrices are diagonalizable by orthogonal matrices $O^TO=\mathbb{I}$ via congrunecy (stated from wiki). $O^TMO = O^{-1}MO = D$ where $D$ is diagonal. 
\end{proof}

\begin{comment}
\begin{lemma}Let A be a real $2 \times 2$ Hamiltonian matrix with eigenvalues $\pm\beta i$, $\beta \neq 0$. Then there exists a real $2 \times 2$ symplectic matrix $S$ such that:
\begin{equation}
S^{-1}AS = \begin{pmatrix} 0 & \beta \\ -\beta & 0 \end{pmatrix} \quad or \quad S^{-1}AS = \begin{pmatrix} 0 & -\beta \\ \beta & 0  \end{pmatrix}
\end{equation} \label{lem:real22}
\end{lemma}
\begin{proof}
See p47 Meyer.
\end{proof}
Possibly not needed.
\end{comment}

%%%%%%%%%%%%%%%% H to A
\begin{lemma}Let H be a real $2\times 2$ symmetric matrix with eigenvalues $\mu_1, \, \mu_2 > 0$ or $\mu_1, \, \mu_2 < 0$. This implies that $A=\Omega H$ has pure imaginary eigenvalues. \label{lem:htoa} \end{lemma} 
\begin{proof}
\begin{equation}
H = \begin{pmatrix} a & b \\ b & c \end{pmatrix}
\end{equation}
Using two eigenvalue identities,
\begin{equation}
\text{det}(M) = \prod_{i=1}^n \lambda_i \quad \text{and} \quad \text{Tr}(M) = \sum_{i=1}^n \lambda_i,
\end{equation}
where $M$ is an $n \times n$ matrix, we have
\begin{equation}
ac-b^2 = \mu_1\mu_2 \quad and \quad a+c = \mu_1+\mu_2.
\end{equation}
Now,
\begin{equation}
A = \begin{pmatrix} 0 & 1 \\ -1 & 0 \end{pmatrix} \begin{pmatrix} a & b \\ b & c \end{pmatrix} = \begin{pmatrix} b & c \\ -a & -b \end{pmatrix}.
\end{equation}
The eigenvalues of A are therefore:
\begin{equation}
\lambda = \pm i\sqrt{\mu_1\mu_2}
\end{equation}
\end{proof}

%%%%%%%%%%%%%%%%%%% A diagonalisable.
\begin{lemma} $H$ is real and symmetric and in diagonal form. If each of its 2-blocks are positive definite, negative definite or 0 then A is diagonalisable. \label{lem:adiag} \end{lemma}
\begin{proof}
Do $A = \Omega H$ so we know that A is block diagonalisable into 2-blocks. We can diagonalise each of these 2-blocks because they fulfil the criteria of lemma (\ref{lem:2n2n}).
\end{proof}