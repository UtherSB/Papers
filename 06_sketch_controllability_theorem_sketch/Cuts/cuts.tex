\begin{lemma}
Let $\rho : \mathcal{L} \to \mathfrak{gl}(V)$ be a finite dimensional representation of a semisimple Lie algebra $\mathcal{L}$. Then $\rho$ is completely reducible. \cite{Humphreys pg28}
\end{lemma}

\begin{definition}[Completely reducible]{$V$ is completely reducible if $V$ is a direct sum of irreducible $\mathcal{L}$-modules$. \cite{Humphreys pg25}}
\end{definition}

\begin{lemma}
$\mathfrak{sp}(2n,\mathbb{R})$ is a split Lie algebra and so finite representations of it are completely reducible into spaces whose complexification is $\eta(+i\beta_j) \oplus \eta(-i\beta_j)$, the subspaces of the representations of the associated semisimple Lie algebra. \cite{??}.
\end{lemma}

\begin{equation}
A = \bigoplus_{j=1}^s A_j \quad \text{where } A_j = JH_j
\end{equation}
where $s$ is the number of A-invariant subspaces. DOUBLE CHECK ALL THIS FOR SPLIT ALGEBRAS.















% Comment
\begin{comment}
\textbf{Rephrasing everything:}\\
\begin{equation}
A = \begin{pmatrix} 0 & B \\ -C & 0 \end{pmatrix}
\end{equation}
A is a pure imaginary, diagonalisable, $2n \times 2n$ Hamiltonian matrix with eigenvalues $\pm i\beta$. $B$ and $C$ are both $n \times n$ diagonal matrices. The spectrum of $A$ contains $n$ lots of $+i\beta$ and $n$ lots of $-i\beta$. $\text{det}(A) = \beta^{2n}$.
\begin{proposition}
$B = \text{diag}(b, b,\ldots)$ and $C = \text{diag}(c, c, \ldots)$ where $b, c \in \mathbb{R}^+$. $\beta = \sqrt{bc}$. 
\end{proposition}
\begin{proof}
To do.
\end{proof}
In this case this would imply that the eigenvalues of $H$ were $b$ and $c$.

The other proposition becomes:
\begin{proposition}
If $B$ and $C$ are both definite in the same sense then A is purely imaginary. An extension of this down to $B$ and $C$ being made up of 2 values of element because haven't got further with the other one. Although this may not be helpful for this proof.
\end{proposition}

\subsection{Lemma for the prop}
To prove above just need:
\begin{equation} M = \begin{pmatrix} 0 & A \\ B & 0 \end{pmatrix} \end{equation}
\begin{lemma}
The eigenvalues of $M$ are $\pm \sqrt{a1b1},\pm \sqrt{a2b2},\ldots$.
\end{lemma}
\begin{proof}

\end{proof}
\end{comment}







\begin{lemma}
$A$ is Hamiltonian implies that $S^(-1)AS = A'$ is Hamiltonian, where $S \in Sp(2n,\mathbb{R} orC)
\end{lemma}
\begin{proof}
See Lin paper on desktop.
\end{proof}








\subsection{H is Diracianly diagonalisable and in SHM form implies that A is pure imaginary and diagonalisable}
\begin{proposition}
If sign($D_{H}(i,i)) = sign(D_{H}(i+n,i+n))\; \forall \; i$ then $A'$ is pure imaginary and diagonalisable.
\end{proposition}
\begin{proof}
The diagonal elements of $D_H$ are $\alpha_i$ and $-\tau_i$. The claim is therefore that if $\alpha_i$ and $\tau_i$ have opposite signs then the eigenvalues of $A'$ are pure imaginary. This can be seen in the equation above. 

The matrix $A'$ is diagonalisable because it is of the form where all the columns only have one element.....So all A is diagable?! SORT THIS OUT.
\end{proof}